\documentclass[12pt]{article}

\usepackage{amsmath,amsthm,amsxtra,amssymb}
\usepackage{rotating}
\usepackage{moreverb}
\usepackage{epsfig}

\newtheorem{frage}{Frage}
\newtheorem{aufg}{Aufgabe}


\begin{document}

   \pagestyle{empty}
   \parindent0cm
    

   \begin{center}
     {\Large \bf Visuelle Kryptografie} \\
    

  \end{center}
   \vspace{0.5cm}
   \normalsize
   \parindent0cm

   
\underline{Was ist Kryptographie?}\\
\\
Eine Nachricht wird so verschl\"usselt, dass sie nur mit dem richtigen Schl\"ussel erkannt werden kann.
So kann zum Beispiel ein Sender eine geheime Nachricht sicher zu einem Empf\"anger senden, ohne dass ein
m\"oglicher Angreifer sie unterwegs erkennen kann.\\

\vspace{2.5cm}
Bei visueller Kryptografie wird ein Schwarz-Weiss-Bild in mehrere Teilbilder codiert, so dass jedes 
Teilbild wie ein zuf\"alliges Punktmuster wirkt. Legt man die Teilbilder dann \"ubereinander, so erkennt 
man das codierte Bild.\\

\vspace{0.5cm}
\underline{Verfahren:}\\
\\
Jeder Bildpunkt wird in 4 Subpixel zerlegt. Bei 2 Teilbildern geht man folgendermassen vor: Jeder 
Bildpunkt wird auf jedem Teilbild mit gleicher Wahrscheinlichkeit durch eine der folgenden Subpixel-Kombinationen 
dargestellt:\\

\vspace{3.5cm}

F\"ur einen weissen Punkt w\"ahlen wir zwei gleiche Kombinationen:

\vspace{3.5cm}

F\"ur einen schwarzen Punkt w\"ahlen wir zwei verschiedene Kombinationen:

\vspace{3.5cm}

Ein Teilbild zeigt uns eine zuf\"allige Verteilung der beiden Subpixel-Kombinationen und liefert daher einem 
m\"oglichen Angreifer keine Information \"uber das codierte Bild. Der Schl\"ussel ist genauso gross wie die verschl\"usselte
Nachricht.\\

\begin{frage}
Ist das Verfahren sicher?
\end{frage}

\begin{frage}
Wann ist das Verfahren sicher?
\end{frage}

\begin{frage}
Wann ist das Verfahren nicht sicher?
\end{frage}

\begin{frage}
Wie kann das Verfahren sicherer gemacht werden, wenn das codierte Bild auf dem Display eines Bankautomaten ist und in unserer
Bankkarte eine Schl\"usselfolie?
\end{frage}

Wird mehr als ein Bild mit derselben Folie verschl\"usselt, so kann man den Inhalt der Bilder erkennen, sobald man 
zwei der Bilder und den Schl\"ussel kennt.

\begin{frage}
Wie? Warum?
\end{frage}

Zum Beispiel wird ein Pixel, der in beiden geheimen Bildern schwarz ist dann weiss:

\vspace{7cm}

\begin{frage}
Was passiert mit einem Pixel, der in beiden Bildern weiss ist?
\end{frage}

\begin{frage}
Was passiert mit einem Pixel, der in einem Bild weiss und im andern schwarz ist?
\end{frage}

Richtig spannend wird visuelle Kryptographie erst, wenn man das geheime Bild auf mehr 
als zwei Folien verteilt. Das unten stehende Beispiel zeigt eine Variante mit drei Folien.\\

\vspace{7cm}

\begin{aufg}
Sucht nach weiteren Varianten mit drei Folien!
\end{aufg}
    
   



\end{document}

   